\chapter{Introdução}

O capítulo de Introdução deve ter no máximo o total de 3 páginas.

% Contextualização e Motivação
\section{Contextualização e Motivação}
\label{sec:context}
% TODO
Nesta seção  descrevem-se a área de  pesquisa na qual o  trabalho está
inserido, o problema e/ou as  circunstâncias que motivaram o projeto e
as potenciais contribuições oriundas de  sua realização. Além disso, é
necessário  sintetizar o  que foi  feito no  Projeto I,  caso o  aluno
esteja  matriculado  atualmente  no  Projeto II.  Se  este  projeto  é
continuidade do anterior, indique isto; se não é, assim mesmo, indique
o que foi o anterior.


%Oi~\citep{silva:12},~\citet{silva:12}.

% Objetivos
\section{Objetivos}
% TODO
Indique claramente quais são  os objetivos do trabalho, caracterizando
de  forma sucinta  o que  se  pretende atingir.  Se o  emprego de  uma
metodologia específica de trabalho for relevante, indique também; caso
contrário, ela aparece apenas na seção de desenvolvimento do trabalho.


% Organização do Trabalho
\section{Organização da Monografia}
% TODO
Descreva a organização do restante da monografia, por exemplo, dizendo
o que o leitor espera encontrar nos próximos capítulos. Indique também
a existência de apêndices e anexos, se houver.


