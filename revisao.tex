\chapter{Métodos, Técnicas e Tecnologias Utilizadas}

Este capítulo não deve ultrapassar o total de 5 páginas.

% Considerações Iniciais
\section{Considerações Iniciais}
% TODO
Neste capítulo  devem ser apresentados  os conceitos e  a terminologia
básicos da área  do projeto e o  levantamento bibliográfico necessário
para  a  realização  do  trabalho.  Em  particular,  a  descrição  dos
principais  trabalhos de  pesquisa  relacionados com  este (em  geral,
aqueles  que representam  o  estado da  arte na  área  de pesquisa  em
questão), bem como  dos trabalhos que serviram de base  para a solução
proposta por este  projeto (em geral, aqueles  que apresentam técnicas
ou  recursos  que  foram  utilizados  pelo  projeto).  A  divisão  nas
subseções   a  seguir   é  opcional.    Em  particular,   nesta  seção
(Considerações Iniciais), descreva sucintamente o que será apresentado
neste capítulo.  As seções a seguir podem ser organizadas da forma que
melhor se adequar à sua monografia.


% Conceitos e Técnicas Relevantes
\section{Conceitos e Técnicas Relevantes}
%TODO



% Trabalhos Relacionados
\section{Trabalhos Relacionados}
%TODO
Refereciar  aqui  os  trabalhos  mais  relacionados  ao  projeto.  Não
esquecer de citar a fonte corretamente, por exemplo \cite{silva:12}.


% Considerações Finais
\section{Considerações Finais}
%TODO
Nesta  seção  deve apresentar  uma  conclusão  sobre este  capítulo  e
introduzir brevemente o capítulo seguinte.
