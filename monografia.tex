\documentclass[12pt,times,a4paper,twoside]{icmc}

\usepackage[brazil]{babel}
\usepackage[utf8]{inputenc}
\usepackage[top=30mm,bottom=20mm,left=30mm,right=20mm,twoside]{geometry}
\usepackage[usenames,dvipsnames]{color}
\usepackage[nottoc]{tocbibind}
\usepackage{fancychap} % Remova se quiser tirar os detalhes do título do capítulo
\usepackage{indentfirst}
\usepackage{setspace}
\usepackage{graphicx}
\usepackage{epstopdf}
\usepackage{amssymb}
\usepackage{amsmath}
\usepackage{mathptmx} % Necessário para corrigir a fonte para Times
\usepackage{hyperref}
\usepackage{setspace}
\usepackage{algpseudocode}
\usepackage{pdfpages}

\hypersetup{
    colorlinks = true,
    citecolor = black,
    filecolor = black,
    linkcolor = black,
    urlcolor = black,
}

\begin{document}
\onehalfspacing

\title{Título}

\author{Autor}

\begingroup
%\maketitle
% CAPA
% Crie a capa no Word (4 páginas iniciais) e exporte para pdf
% \includepdf[pages={1-4}]{caminho-para-capa.pdf}

\frontmatter \pagestyle{plain}

\floatplacement{table}{!ht}

% Epígrafe
\begin{titlepage}
\vspace*{20cm}
\begin{flushright}
\begin{minipage}[t]{7.0 cm}
\textit{Texto para epígrafe aqui.\\Obs.: item opcional.}
\end{minipage}
\end{flushright}
\end{titlepage}

% Dedicatória e Agradecimentos
\chapter*{Dedicatória}
\begin{flushright}
    Dedicatória aqui.\\
    Obs.:Item opcional.
\end{flushright}

\chapter*{Agradecimentos}

Agradecimentos aqui.\\
Obs.:Item opcional.
%%%%%%%%%%%%%%%%%%%%%%%%%%

\input resumo.tex

\tableofcontents

\listoffigures
\listoftables
\input siglas.tex
\input simbolos.tex

\cleardoublepage{\pagestyle{plain}\clearpage}

\endgroup

\mainmatter

\renewcommand{\chaptermark}[1]{%

\markboth{\chaptername
\ \thechapter.\ #1}{}}  %

\renewcommand{\sectionmark}[1]{%
 \markright{\thesection.\ #1}}

%\input ...
\input introducao.tex
\input revisao.tex
\input desenvolvimento.tex
\input conclusao.tex

%Observação 1: É obrigatório que a monografia tenha uma lista de referências que deve estar contida neste item.
%Observação 2: As referências devem estar em ordem alfabética pelo sobrenome do primeiro autor.
%Observação 3: Todos os documentos referenciados nesse item (livros, artigos, relatórios técnicos, sites, entre outros) deverão ter sido citados no texto.
%Observação 4: Todos as citações no decorrer do texto deverão ser listadas neste item.
%Observação 5: É obrigatório que a lista de referências esteja de acordo com a norma NBR-6023/2002 para referências bibliográficas.

\renewcommand{\bibname}{Referências}
%\printbibliography[heading=bibintoc]
\bibliographystyle{abntex2-alf} 
\bibliography{referencias}

% Se precisar de apêndice, use a seção abaixo
%\appendix
%\renewcommand\appendixname{Apêndice}
%\renewcommand\chaptername{Apêndice}

\end{document}
