\chapter{Desenvolvimento do Trabalho}

Este capítulo não deve ultrapassar o total de 9 páginas.

% Considerações Iniciais
\section{Considerações Iniciais}
%TODO
Neste capítulo o projeto deve ser detalhadamente descrito, de modo que
o leitor identifique todos os  passos da metodologia adotada, bem como
todos os  recursos e técnicas utilizados.  Além disso, e quando  for o
caso, os resultados e sua  avaliação devem ser descritos e analisados.
Em particular, nesta seção (Considerações Iniciais), deve-se descrever
sucintamente  o que  será  apresentado neste  capítulo.   As seções  a
seguir  podem ser  organizadas da  forma que  melhor se  adeque à  sua
monografia, contudo, são sugeridas as seguintes.

% Projeto
\section{Descrição do Problema}
%TODO

Nesta  seção descreva  os  objetivos  do trabalho,  sua  proposta e  a
metodologia geral  para seu desenvolvimento (por  exemplo, apresente a
arquitetura do sistema  que foi implementado, descrevendo  a função de
cada um de  seus módulos; ou apresente todos os  passos de um processo
que deverá ser  executado). O detalhamento do  trabalho executado (por
exemplo, como foi  implementado cada módulo do sistema,  ou cada etapa
de um processo) deve ser feito na seção seguinte.

% Atividades Realizadas
\section{Descrição das Atividades Realizadas}
%TODO
Nesta  seção  o  aluno  deve  descrever  em  detalhes  cada  etapa  da
metodologia  descrita   na  seção   anterior.  Identifique   todos  os
recursos/técnicas/sistemas   utilizados  em   cada  etapa.   Pode  ser
necessário criar  várias subseções  para acomodar todas  as atividades
realizadas.

% Avaliação dos Resultados
\section{Análise e Avaliação dos Resultados}
%TODO
Nesta seção, descreva o processo  de obtenção dos resultados obtidos e
analise-os  à  luz dos  objetivos  iniciais,  bem como  dos  eventuais
usuários  (humanos  ou  não) de  tais  resultados.  Preferencialmente,
utilize   critérios  estabelecidos   na  área   de  pesquisa   de  seu
projeto.  Discuta  sobre  critérios  utilizados para  a  validação  do
sistema, bem como os testes utilizados (se for o caso).

% Dificuldades, Limitações e Trabalhos Futuros
\section{Dificuldades, Limitações e Trabalhos Futuros}
%TODO
Nesta  seção,   descreva  as  principais  dificuldades   e  limitações
encontradas  durante   a  condução   do  trabalho.   Sintetize  lições
aprendidas e  comente sobre direções  alternativas, se for o  caso. Se
pertinente,  faça uma  análise  crítica da  abordagem  adotada em  seu
projeto, ou seja, você a considera  adequada? Ela é limitada sob algum
aspecto?  Indique quais  pontos, decorrentes  dos resultados,  ou não,
poderão ou deverão ser abordados por trabalhos futuros.

% Considerações Finais
\section{Considerações Finais}
Nesta seção o aluno deve  apresentar uma conclusão sobre este capítulo
e introduzir brevemente o capítulo seguinte.
